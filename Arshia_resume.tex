%%%%%%%%%%%%%%%%%%%%%%%%%%%%%%%%%%%%%%%%%
% Twenty Seconds Resume/CV
% LaTeX Template
% Version 1.0 (14/7/16)
%
% Original author:
% Carmine Spagnuolo (cspagnuolo@unisa.it) with major modifications by 
% Vel (vel@LaTeXTemplates.com) and Harsh (harsh.gadgil@gmail.com)
%
% License:
% The MIT License (see included LICENSE file)
%
%%%%%%%%%%%%%%%%%%%%%%%%%%%%%%%%%%%%%%%%%

%----------------------------------------------------------------------------------------
%	PACKAGES AND OTHER DOCUMENT CONFIGURATIONS
%----------------------------------------------------------------------------------------

\documentclass[letterpaper]{twentysecondcv} % a4paper for A4

% Command for printing skill overview bubbles
\newcommand\skills{ 
	~
	\smartdiagram[bubble diagram]{
		\textbf{Artificial }\\\textbf{Intelligence},
		\textbf{Deep}\\\textbf{Learning},
		\textbf{Feature}\\\textbf{Engineering},
		\textbf{Reinforcement}\\\textbf{Learning},
		\textbf{Machine}\\\textbf{Learning},
		\textbf{Data}\\\textbf{~~Analysis~~},
		\textbf{Statistical}\\\textbf{Analysis}
	}
}

% Programming skill bars
\programmingandlibraries{{OpenCV$\textbullet$ Chainer $\textbullet$ Caffe / 3}, {CNTK$\textbullet$Theano $\textbullet$ Pytorch/ 4.5}, {Python$\textbullet$Tensorflow$\textbullet$Keras$\textbullet$Scikit-learn / 5}}

% Projects text
\education{
\textbf{MSc., Computer Science} (GPA: 3.4) \\
Specialization: Machine Learning \\
The University of Winnipeg \\
2014 - 2017 | Winnipeg,Manitoba,Canada

\textbf{BEng.,Electronics\&Communication} (GPA: 3.6) \\
Karpagam University \\
2009 - 2012 | TamilNadu, India
}

%----------------------------------------------------------------------------------------
%	 PERSONAL INFORMATION
%----------------------------------------------------------------------------------------
% If you don't need one or more of the below, just remove the content leaving the command, e.g. \cvnumberphone{}

\cvname{Arshia Sathya \\ Ulaganathan} % Your name
\cvjobtitle{ Machine Learning Engineer } % Job
% title/career

\cvlinkedin{/in/arshiau/}
\cvgithub{Arshiasathya}
\cvnumberphone{(204) 990 7543} % Phone number
\cvmail{arshiasathya@gmail.com} % Email address

%----------------------------------------------------------------------------------------

\begin{document}
\makeprofile % Print the sidebar


%----------------------------------------------------------------------------------------
%Interest
%----------------------------------------------------------------------------------------
\vspace{-0.2cm}
\section{Interests}
\vspace{-0.2cm}
	{Self motivated, hard working, experienced machine learning engineer. Interested in exploring new dimensions of AI application and solving the real world problems using the cutting edge technology in AI.}
 \vspace{-0.2cm}
%----------------------------------------------------------------------------------------
%	 EXPERIENCE
%----------------------------------------------------------------------------------------
 \vspace{-0.2cm}
\section{Experience}
 \vspace{-0.2cm}
\begin{twenty} % Environment for a list with descriptions
\twentyitem
    	{\bf April 2018 - }
		{\bf Present}
        {\large  Jr.Machine Learning Engineer}
        {\href{https://laivly.com/}{\bf \large Lavily/24-7 Intouch}}
        {\textbf{Projects: Email template recommendation engine}}
        {\begin{itemize}
       	\vspace{-0.4cm}
        \item Tool that assists customer care agents to select the proper template for response by analyzing customer emails.
        \item The goal of project is to reduce the average handling time ticket and improve the customer satisfaction.\\ 
        \textbf{Conversational AI assistant:}
        \item Conversational AI assistant helps the customer care agents handle multiple chats efficiently.The purpose of this project is to increase the concurrency of the chat per agent
        \item Automated generic chat flow with the indication for human assistance when the bot is in need.\\
        \textbf{Intent and Sentiment classification:}
        \item Narrow's down the path for template sugesstion for response
        \item Sentiment of the email or message is used to select the responses, this helps to add proper verbage in the responses.\\
        \textbf{Voice emotion detection(PoC stage):}
        \item This project tracks how sentiment varies thorough out the call. 
        \item Helps team leads find out when and which agent needs help from them to handle critical customer calls.
        \item \textbf{Langugage and Libraries used:}Python, Jupyter Notebook, Tensorflow, Keras, PyTorch, Scikit-learn, Numpy, Pandas, PyAudioAnalysis, Librosa, Matplotlib, Seaborn, Scipy, Rasa, Flask. 
        \item \textbf{Algorithms or Models used:}CNN, LSTM, RNN, GRU, Transformer, Encoder Decoder, Capsule Network, Siamese n shot learning.\\
        \textbf{Software development and DevOps Experiences:}
        \item Experienced in Agile methodology, Bitbucket as version control.
        \item Experienced with Amazon web service(aws) Kubernetes, Docker cloud sources
        \vspace{-0.2cm}
   	    \end{itemize}}
        \\
        \vspace{-0.2cm}
    \twentyitem
   		{\bf Jan 2016 -}
		{\bf Apr 2016}
        {\large Graduate Teaching Assistant }
        {\href{https://www.uwinnipeg.ca/}{\bf \large The University of Winnipeg}}
        {}
        {
        {\begin{itemize}
        \item Created tutorial materials for software project management course
        \item Marked assignments and exams
        \item Invigilated exams
        \vspace{-0.4cm}
    \end{itemize}}
        }
     \\
     \twentyitem
   		{\bf May 2015 -}
		{\bf Apr 2018}
        {\large Sales Associate}
        {\href{https://www.homedepot.ca/en/home.html}{\bf \large The Home Depot}}
        {}
        {
        \begin{itemize}
        	\vspace{-0.2cm}
        \item Inventory Assessment
        \item Product Knowledge 
        \item Assited customers with purchase 
        \item Point of sale transactions
    \end{itemize}
    	}
        
	%\twentyitem{<dates>}{<title>}{<location>}{<description>}
\end{twenty}
 \vspace{-0.2cm}
%----------------------------------------------------------------------------------------
%	 RESEARCH
%----------------------------------------------------------------------------------------
\vspace{-0.2cm}
\section{Research}
\vspace{-0.1cm}
\begin{twenty}
	\twentyitem
    	{\bf 2015 -}
    	{\bf 2017}
        {\large MSc. Candidate, Graduate Research Assistant}
        {\href{https://www.uwinnipeg.ca/}{\bf \large The University of Winnipeg}}
        {}
        {
       	\textbf{Thesis}: Granular Methods in Automatic Music Genre Classification: A Case Study
        {\begin{itemize}
        \item Proposed and implemented \textbf{algorithm }Tolerance Class Learner (TCL)2.0 for classification.
        \item Worked on feature selection and feature reduction methods for classification
        \item \textbf{Tools}: R, Python, scikit-learn, pandas, numpy. \vspace{2mm}
		\end{itemize}}
        }
\end{twenty}
\vspace{-0.6cm}
\section{Publications}
\vspace{-0.2cm}
Ulaganathan, Arshia Sathya, Ramanna, “Granular Methods in Automatic Music Genre Classification: A Case Study” in 2019  Journal of Intelligent Information Systems, pp. 85-105, J. Intell. Inf. Syst., 2019. \vspace{2mm}
\vspace{-0.6cm}
\end{document} 
